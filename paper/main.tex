\documentclass{article}
% packages
\usepackage{amsmath,amssymb}
\usepackage{graphicx}
\usepackage{hyperref}

% directory of figures
%\graphicspath{ {figs} }

% latin bold lower
\newcommand{\ba}{\mathbf{a}} 
\newcommand{\bc}{\mathbf{c}} 
\newcommand{\be}{\mathbf{e}} 
\newcommand{\bh}{\mathbf{h}} 
\newcommand{\bp}{\mathbf{p}} 
\newcommand{\bt}{\mathbf{t}} 
\newcommand{\bs}{\mathbf{s}} 
\newcommand{\bu}{\mathbf{u}} 
\newcommand{\bv}{\mathbf{v}} 
\newcommand{\bw}{\mathbf{w}} 
\newcommand{\bx}{\mathbf{x}} 
\newcommand{\by}{\mathbf{y}} 
\newcommand{\bz}{\mathbf{z}} 
\newcommand{\bm}{\mathbf{m}} 

% latin bold upper
\newcommand{\bA}{\mathbf{A}} 
\newcommand{\bB}{\mathbf{B}} 
\newcommand{\bC}{\mathbf{C}} 
\newcommand{\bI}{\mathbf{I}} 
\newcommand{\bJ}{\mathbf{J}} 
\newcommand{\bL}{\mathbf{L}} 
\newcommand{\bM}{\mathbf{M}} 
\newcommand{\bP}{\mathbf{P}}
\newcommand{\bQ}{\mathbf{Q}} 
\newcommand{\bR}{\mathbf{R}} 
\newcommand{\bT}{\mathbf{T}} 
\newcommand{\bU}{\mathbf{U}} 
\newcommand{\bV}{\mathbf{V}} 
\newcommand{\bW}{\mathbf{W}} 
\newcommand{\bX}{\mathbf{X}} 
\newcommand{\bY}{\mathbf{Y}} 
\newcommand{\bZ}{\mathbf{Z}} 

% latin cal upper
\newcommand{\cF}{\mathcal{F}} 
\newcommand{\cG}{\mathcal{G}} 
\newcommand{\cI}{\mathcal{I}} 
\newcommand{\cL}{\mathcal{L}} 
\newcommand{\cM}{\mathcal{M}} 
\newcommand{\cN}{\mathcal{N}} 
\newcommand{\cS}{\mathcal{S}} 
\newcommand{\cT}{\mathcal{T}} 
\newcommand{\cW}{\mathcal{W}} 
\newcommand{\cX}{\mathcal{X}} 
\newcommand{\cZ}{\mathcal{Z}} 

% latin bb upper
\newcommand{\bbE}{\mathbb{E}} 
\newcommand{\bbI}{\mathbb{I}} 
\newcommand{\bbP}{\mathbb{P}} 
\newcommand{\bbR}{\mathbb{R}}
\newcommand{\bbX}{\mathbb{X}} 
\newcommand{\bbY}{\mathbb{Y}}
\newcommand{\bbW}{\mathbb{W}} 

% greek bold lower
\newcommand{\bepsilon}{\boldsymbol{\epsilon}} 
\newcommand{\btheta}{\boldsymbol{\theta}} 
\newcommand{\blambda}{\boldsymbol{\lambda}} 
\newcommand{\bpi}{\boldsymbol{\pi}} 
\newcommand{\bmu}{\boldsymbol{\mu}} 
\newcommand{\bsigma}{\boldsymbol{\sigma}} 
\newcommand{\bphi}{\boldsymbol{\phi}} 

% greek bold upper
\newcommand{\bSigma}{\boldsymbol{\Sigma}} 

\DeclareMathOperator*{\argmin}{arg\,min}
\DeclareMathOperator*{\argmax}{arg\,max}

% transpose
\newcommand{\T}{^{\text{\tiny\sffamily\upshape\mdseries T}}}

% if you need to pass options to natbib, use, e.g.:
\PassOptionsToPackage{numbers, sort, compress}{natbib}
% before loading neurips_2024


% ready for submission
%\usepackage{neurips_2024}


% to compile a preprint version, e.g., for submission to arXiv, add add the
% [preprint] option:
\usepackage[preprint]{neurips_2024}


% to compile a camera-ready version, add the [final] option, e.g.:
%     \usepackage[final]{neurips_2024}


% to avoid loading the natbib package, add option nonatbib:
%    \usepackage[nonatbib]{neurips_2024}


\usepackage[utf8]{inputenc} % allow utf-8 input
\usepackage[T1]{fontenc}    % use 8-bit T1 fonts
\usepackage{hyperref}       % hyperlinks
\usepackage{url}            % simple URL typesetting
\usepackage{booktabs}       % professional-quality tables
\usepackage{amsfonts}       % blackboard math symbols
\usepackage{nicefrac}       % compact symbols for 1/2, etc.
\usepackage{microtype}      % microtypography
\usepackage{xcolor}         % colors

%%%

\usepackage{subcaption}
\usepackage{graphicx}
\usepackage{multirow}
\usepackage{amsmath,amssymb,amsfonts}
\usepackage{amsthm}
\usepackage{mathrsfs}
\usepackage{xcolor}
\usepackage{textcomp}
\usepackage{manyfoot}
\usepackage{booktabs}
\usepackage{algorithm}
\usepackage{algorithmicx}
\usepackage{algpseudocode}
\usepackage{listings}

\newtheorem{theorem}{Theorem} % continuous numbers
%%\newtheorem{theorem}{Theorem}[section] % sectionwise numbers
%% optional argument [theorem] produces theorem numbering sequence instead of independent numbers for Proposition
\newtheorem{proposition}[theorem]{Proposition}% 
\newtheorem{lemma}{Lemma}% 
%%\newtheorem{proposition}{Proposition} % to get separate numbers for theorem and proposition etc.

\newtheorem{example}{Example}
\newtheorem{remark}{Remark}

\newtheorem{definition}{Definition}
\newtheorem{assumption}{Assumption}

%%%


\title{Detecting Manual Alterations in Biological Image Data 
Using Contrastive Learning and Pairwise Image Comparison}


% The \author macro works with any number of authors. There are two commands
% used to separate the names and addresses of multiple authors: \And and \AND.
%
% Using \And between authors leaves it to LaTeX to determine where to break the
% lines. Using \AND forces a line break at that point. So, if LaTeX puts 3 of 4
% authors names on the first line, and the last on the second line, try using
% \AND instead of \And before the third author name.


\author{%
  Georgii Nekhoroshkov\\
  MIPT\\
  Moscow, Russia\\
  \texttt{nekhoroshkov.gs@phystech.edu}\\
  \And
  Daniil Dorin\\
  MIPT\\
  Moscow, Russia\\
  \texttt{dorin.dd.contact@gmail.com}\\
  \And
  Andrii Hraboviy\\
  MIPT\\
  Moscow, Russia\\
  \texttt{grabovoy.av@phystech.edu}\\
}


\begin{document}


\maketitle

\begin{abstract}

    In this paper, we address the problem of detecting manipulations in biological images. 
    Ensuring the integrity of biological 
    image data is essential for reliable scientific research. 
    The study focuses on developing a model for pairwise image comparison
    using contrastive learning, demonstrating high pairwise comparison metrics to detect 
    manual modifications or more subtle alterations. 
    The proposed method outperforms state-of-the-art models, 
    including CLIP and Barlow Twins, in the task of biological 
    image comparison on fMRI scans and cell datasets. 
    This work contributes to automated fraud detection and data validation in 
    biological research.

\end{abstract}

\textbf{Keywords:}
Machine Learning, Pairwise Image Comparison, Self-Supervised Learning, 
Fine-Tuning, Automated Fraud Detection, Detecting Data Alterations

\section{Introduction}\label{sec:intro}

Our work aims to develop a machine learning solution for the problem 
of reusing existing biological and medical snapshots to demonstrate results 
in newly published biological articles. Fake images negatively impact on 
medicine by providing false or fabricated results and undermining the credibility 
of new scientific work in these fields. Existing state-of-the-art self-supervised learning 
approaches demonstrate remarkable results in pairwise image comparison tasks 
(\texttt{SimCLR} \cite{chen2020simclr}, \texttt{CLIP} \cite{radford2021clip}, 
\text{Barlow Twins} \cite{zbontar2021barlow}). 
However, their performance significantly worsens when applied to complex biological data. 
It requires developing model that is more sensitive to subtle changes in the image content 
while remaining resistant to various manual alterations, such as color jittering, 
flipping, rotation, noise application, and random affine transformations. 
At present, the problem of matching biological and medical images remains unsolved due 
to the complexity of distinguishing snapshots of similar objects, where 
differences can only be identified by experts in the field.

We propose a solution, based on \texttt{Barlow Twins} \cite{zbontar2021barlow},
trained and fine-tuned specifically for complex biological scans. 
The model belongs to the family of self-supervised learning (SSL) 
methods, which have been proven to be competitive with supervised representations 
(\cite{chen2020simclr}, \cite{radford2021clip}, \cite{zbontar2021barlow}, \cite{melekhov2016}, 
\cite{grill2020approach_ssl}). By leveraging a pretrained model, 
it does not require large snapshot datasets to achieve state-of-the-art accuracy 
on the \texttt{Haxby fMRI}, \texttt{CIL Epithelial Cell}, 
\texttt{CIL Lymphocyte Cell} datasets. This solution can be widely used by 
biological articles proofreaders to verify the authenticity of provided images and 
detect borrowings from known datasets. 

%\textbf{Contributions.} Our contributions can be summarized as follows:
%\begin{itemize}
%    \item We present...
%    \item We demonstrate the validity of our theoretical results through empirical studies...
%    \item We highlight the implications of our findings for...
%\end{itemize}

%\textbf{Outline.} The rest of the paper is organized as follows...

\section{Problem}\label{sec:problem}

Given dataset $\mathcal{D}$, consisting of $N$ biological snapshots: 

$$ \mathcal{D} = \{d_i \in [0, 256)^{l \times l \times 3}, i \in [0, N)\} $$

where $d_i$ is the RGB-decoded image, $l$ -- the length of an image side. 

For simplicity, we will refer to a pair of images with the same content 
before alterations as a \textit{similar} pair; otherwise, it will be called \textit{dissimilar}.
Our goal is to build a model $\mathcal{M}$ using self-supervised contrastive learning (SSCL), 
which should be able to distinguish dissimilar pairs of images and identify similar ones. 

Let $x$ and $y$ be two images ($x, y \in [0, 256)^{l \times l \times 3}$).
The model consists of two main parts:

$$ \mathcal{M}(x, y) = h(f_{\theta}(x), f_{\theta}(y))) $$

where $f_{\theta}$ is an encoder with a trainable parameter set $\theta$, and 
$h$ is the linear classifier:

$$ f_{\theta}(x) = v_x \in \mathbb{R}^{d} $$

$$ h(v_x, v_y) = s \in \{0, 1\} $$

Value $s = 1$ corresponds to similar pairs, $s = 0$ represents dissimilar pairs.

To train the encoder, let us define the loss function $\mathcal{L}$:

$$ \mathcal{L}(v_x, v_y, I(x, y)) \in [0, +\infty) $$

where $v_x$ and $v_y$ are the embeddings of images $x$ and $y$, respectively, 
and $I(x, y)$ is defined as follows:

$$ I(x, y) = \begin{cases} 
1, \text{if $x$ and $y$ are similar}, \\
0, \text{otherwise}
\end{cases} $$

The model's accuracy will be evaluated by counting the number of correctly classified similar pairs 
and incorrectly classified ones, producing a single accuracy value to compare with other 
state-of-the-art models.

\section{Method}

DRAFT

The problem of detecting reused biological and medical images is rooted in the complexity of
distinguishing between similar images while remaining invariant to various transformations.
The dataset consists of biological snapshots from publicly available sources, we will 
store them as an array $D$ with elements $d_i \in [0, 256)^{l \times l \times 3}$, where 
$l$ is the length of images sides. 
Let us define the \textit{solution} -- a method which meant to solve the stated problem.
Also we will use the termini \textit{model} -- the machine learning solution, which is 
proposed in our work.

Structure of the model is the same as in \texttt{Barlow Twins} \cite{zbontar2021barlow}.
It takes a batch of images $X$, each image $x_i$ is being augmented in two ways, producing 
two modified pictures $y_i^A$ and $y_i^B$ for each original. 
The resulted two batches of distorted images $Y^A$ and $Y^B$ are then
fed to a deep network function $f_{\theta}$, and the produced embeddings $Z^A$ and $Z^B$ 
are used to evaluate loss function $\mathcal{L}$ with the following formula, proposed in 
\texttt{Barlow Twins} \cite{zbontar2021barlow}:

$$ \mathcal{L}(Z^A, Z^B) = \sum_i {(1 - \zeta_{ii})^2} + \lambda \sum_{i}\sum_{j \neq i}{\zeta_{ij}^2} $$

where $\lambda$ is the positive constant, and $\zeta_{ij}$ -- cross-correlation matrix, 
computed between the outputs of the two identical networks along the batch dimension:

$$ \zeta_{ij} = 
\frac{\sum_b{z_{b,i}^A z_{b,j}^B}}{\sqrt{\sum_b{(z_{b,i}^A)^2}}\sqrt{\sum_b{(z_{b,j}^B)^2}}} $$

For the further simplicity, we will refer to a pair of images with the same content 
before alterations as a \textit{similar} pair, otherwise it will be called \textit{dissimilar}. 
The idea behind this loss function is to minimize the distance between similar pairs embeddings 
and to make it bigger for dissimilar.

The model is trained on the $70\%$ dataset $D$ data, then validated on another $10\%$ 
and tested on the remaining $20\%$ -- let $D_{test}$ be this part. For accuracy evaluation, 
we follow the widely used linear evaluation protocol (here comes the links from \cite{chen2020simclr}).

\section{Related Work}\label{sec:rw}

\textbf{Topic \#1.}
TODO

\textbf{Topic \#2.}
TODO

\section{Preliminaries}\label{sec:prelim}

\subsection{General notation}

In this section, we introduce the general notation used in the rest of the paper and the basic assumptions. 

\subsection{Assumptions} 

TODO

\section{Experiments}\label{sec:exp}

To verify the theoretical estimates obtained, we conducted a detailed empirical study...

\section{Discussion}\label{sec:disc}

TODO

\section{Conclusion}\label{sec:concl}

TODO


%%%%%%%%%%%%%%%%%%%%%%%%%%%%%%%%%%%%%%%%%%%%%%%%%%%%%%%%%%%%

\bibliographystyle{unsrtnat}
\bibliography{references}

%%%%%%%%%%%%%%%%%%%%%%%%%%%%%%%%%%%%%%%%%%%%%%%%%%%%%%%%%%%%

\newpage
\appendix
\section{Appendix / supplemental material}\label{app}

\subsection{Additional experiments}\label{app:exp}

TODO

\end{document}
