\documentclass{article}
% packages
\usepackage{amsmath,amssymb}
\usepackage{graphicx}
\usepackage{hyperref}

% directory of figures
%\graphicspath{ {figs} }

% latin bold lower
\newcommand{\ba}{\mathbf{a}} 
\newcommand{\bc}{\mathbf{c}} 
\newcommand{\be}{\mathbf{e}} 
\newcommand{\bh}{\mathbf{h}} 
\newcommand{\bp}{\mathbf{p}} 
\newcommand{\bt}{\mathbf{t}} 
\newcommand{\bs}{\mathbf{s}} 
\newcommand{\bu}{\mathbf{u}} 
\newcommand{\bv}{\mathbf{v}} 
\newcommand{\bw}{\mathbf{w}} 
\newcommand{\bx}{\mathbf{x}} 
\newcommand{\by}{\mathbf{y}} 
\newcommand{\bz}{\mathbf{z}} 
\newcommand{\bm}{\mathbf{m}} 

% latin bold upper
\newcommand{\bA}{\mathbf{A}} 
\newcommand{\bB}{\mathbf{B}} 
\newcommand{\bC}{\mathbf{C}} 
\newcommand{\bI}{\mathbf{I}} 
\newcommand{\bJ}{\mathbf{J}} 
\newcommand{\bL}{\mathbf{L}} 
\newcommand{\bM}{\mathbf{M}} 
\newcommand{\bP}{\mathbf{P}}
\newcommand{\bQ}{\mathbf{Q}} 
\newcommand{\bR}{\mathbf{R}} 
\newcommand{\bT}{\mathbf{T}} 
\newcommand{\bU}{\mathbf{U}} 
\newcommand{\bV}{\mathbf{V}} 
\newcommand{\bW}{\mathbf{W}} 
\newcommand{\bX}{\mathbf{X}} 
\newcommand{\bY}{\mathbf{Y}} 
\newcommand{\bZ}{\mathbf{Z}} 

% latin cal upper
\newcommand{\cF}{\mathcal{F}} 
\newcommand{\cG}{\mathcal{G}} 
\newcommand{\cI}{\mathcal{I}} 
\newcommand{\cL}{\mathcal{L}} 
\newcommand{\cM}{\mathcal{M}} 
\newcommand{\cN}{\mathcal{N}} 
\newcommand{\cS}{\mathcal{S}} 
\newcommand{\cT}{\mathcal{T}} 
\newcommand{\cW}{\mathcal{W}} 
\newcommand{\cX}{\mathcal{X}} 
\newcommand{\cZ}{\mathcal{Z}} 

% latin bb upper
\newcommand{\bbE}{\mathbb{E}} 
\newcommand{\bbI}{\mathbb{I}} 
\newcommand{\bbP}{\mathbb{P}} 
\newcommand{\bbR}{\mathbb{R}}
\newcommand{\bbX}{\mathbb{X}} 
\newcommand{\bbY}{\mathbb{Y}}
\newcommand{\bbW}{\mathbb{W}} 

% greek bold lower
\newcommand{\bepsilon}{\boldsymbol{\epsilon}} 
\newcommand{\btheta}{\boldsymbol{\theta}} 
\newcommand{\blambda}{\boldsymbol{\lambda}} 
\newcommand{\bpi}{\boldsymbol{\pi}} 
\newcommand{\bmu}{\boldsymbol{\mu}} 
\newcommand{\bsigma}{\boldsymbol{\sigma}} 
\newcommand{\bphi}{\boldsymbol{\phi}} 

% greek bold upper
\newcommand{\bSigma}{\boldsymbol{\Sigma}} 

\DeclareMathOperator*{\argmin}{arg\,min}
\DeclareMathOperator*{\argmax}{arg\,max}

% transpose
\newcommand{\T}{^{\text{\tiny\sffamily\upshape\mdseries T}}}

% if you need to pass options to natbib, use, e.g.:
\PassOptionsToPackage{numbers, sort, compress}{natbib}
% before loading neurips_2024


% ready for submission
%\usepackage{neurips_2024}


% to compile a preprint version, e.g., for submission to arXiv, add add the
% [preprint] option:
\usepackage[preprint]{neurips_2024}


% to compile a camera-ready version, add the [final] option, e.g.:
%     \usepackage[final]{neurips_2024}


% to avoid loading the natbib package, add option nonatbib:
%    \usepackage[nonatbib]{neurips_2024}


\usepackage[utf8]{inputenc} % allow utf-8 input
\usepackage[T1]{fontenc}    % use 8-bit T1 fonts
\usepackage{hyperref}       % hyperlinks
\usepackage{url}            % simple URL typesetting
\usepackage{booktabs}       % professional-quality tables
\usepackage{amsfonts}       % blackboard math symbols
\usepackage{nicefrac}       % compact symbols for 1/2, etc.
\usepackage{microtype}      % microtypography
\usepackage{xcolor}         % colors

%%%

\usepackage{subcaption}
\usepackage{graphicx}
\usepackage{multirow}
\usepackage{amsmath,amssymb,amsfonts}
\usepackage{amsthm}
\usepackage{mathrsfs}
\usepackage{xcolor}
\usepackage{textcomp}
\usepackage{manyfoot}
\usepackage{booktabs}
\usepackage{algorithm}
\usepackage{algorithmicx}
\usepackage{algpseudocode}
\usepackage{listings}

\newtheorem{theorem}{Theorem} % continuous numbers
%%\newtheorem{theorem}{Theorem}[section] % sectionwise numbers
%% optional argument [theorem] produces theorem numbering sequence instead of independent numbers for Proposition
\newtheorem{proposition}[theorem]{Proposition}% 
\newtheorem{lemma}{Lemma}% 
%%\newtheorem{proposition}{Proposition} % to get separate numbers for theorem and proposition etc.

\newtheorem{example}{Example}
\newtheorem{remark}{Remark}

\newtheorem{definition}{Definition}
\newtheorem{assumption}{Assumption}

%%%


\title{Detecting Manual Alterations in Biological Image Data 
Using Contrastive Learning and Pairwise Image Comparison}


% The \author macro works with any number of authors. There are two commands
% used to separate the names and addresses of multiple authors: \And and \AND.
%
% Using \And between authors leaves it to LaTeX to determine where to break the
% lines. Using \AND forces a line break at that point. So, if LaTeX puts 3 of 4
% authors names on the first line, and the last on the second line, try using
% \AND instead of \And before the third author name.


\author{%
  Daniil Dorin\\
  MIPT\\
  Moscow, Russia\\
  \texttt{dorin.dd.contact@gmail.com}\\
  \And
  Georgii Nekhoroshkov\\
  MIPT\\
  Moscow, Russia\\
  \texttt{nekhoroshkov.gs@phystech.edu}\\
  \And
  Andrii Hraboviy\\
  MIPT\\
  Moscow, Russia\\
  \texttt{grabovoy.av@phystech.edu}\\
}


\begin{document}


\maketitle

\begin{abstract}

    In this paper, we address the problem of detecting manipulations in biological images. 
    Ensuring the integrity of biological 
    image data is essential for reliable scientific research. 
    The study focuses on developing a model for pairwise image comparison
    using contrastive learning, demonstrating high pairwise comparison metrics to detect 
    manual modifications or more subtle alterations. 
    The proposed method outperforms state-of-the-art models, 
    including CLIP and Barlow Twins, in the task of biological 
    image comparison on fMRI scans and cell datasets. 
    This work contributes to automated fraud detection and data validation in 
    biological research.

\end{abstract}

\textbf{Keywords:}
Machine Learning, Pairwise Image Comparison, Fine-Tuning, Automated Fraud Detection, 
Detecting Data Alterations

\section{Introduction}\label{sec:intro}

TODO

\textbf{Contributions.} Our contributions can be summarized as follows:
\begin{itemize}
    \item We present...
    \item We demonstrate the validity of our theoretical results through empirical studies...
    \item We highlight the implications of our findings for...
\end{itemize}

\textbf{Outline.} The rest of the paper is organized as follows...

\section{Related Work}\label{sec:rw}

\textbf{Topic \#1.}
TODO

\textbf{Topic \#2.}
TODO

\section{Preliminaries}\label{sec:prelim}

\subsection{General notation}

In this section, we introduce the general notation used in the rest of the paper and the basic assumptions. 

\subsection{Assumptions} 

TODO

\section{Experiments}\label{sec:exp}

To verify the theoretical estimates obtained, we conducted a detailed empirical study...

\section{Discussion}\label{sec:disc}

TODO

\section{Conclusion}\label{sec:concl}

TODO


%%%%%%%%%%%%%%%%%%%%%%%%%%%%%%%%%%%%%%%%%%%%%%%%%%%%%%%%%%%%

\bibliographystyle{unsrtnat}
\bibliography{references}

%%%%%%%%%%%%%%%%%%%%%%%%%%%%%%%%%%%%%%%%%%%%%%%%%%%%%%%%%%%%

\newpage
\appendix
\section{Appendix / supplemental material}\label{app}

\subsection{Additional experiments}\label{app:exp}

TODO

\end{document}
